\section{Wykład 1 - wprowadzenie $\heartsuit$ }

\paragraph{Podstawowe cechy sieci neuronowych}

\begin{enumerate}
 \item adaptacja i samoorganizacja
 \item zmniejszona wrażliwość na uszkodzenia elementów
 \item równoległa praca
 \item wygoda programowania przez uczenie
\end{enumerate}

\paragraph{Zastosowania}

\begin{enumerate}
 \item klasyfikacja obrazów
 \item klasteryzacja / kategoryzacja
 \item aproksymacja funkcji 
 \item predykcja / prognozowanie
 \item optymalizacja
 \item odtwarzanie
 \item sterowanie
\end{enumerate}

\paragraph{Przepływ sygnału w NN}

\begin{enumerate}
 \item warstwa wejściowa
 \item warstwa ukryta
 \item warstwa wyjściowa
\end{enumerate}

Warstwa wejściowa nie może się uczyć. (jak jest warstwa wejściowa, 2 ukryte i wyjściowa to jest 
to sieć trójwarstwowa (wejściowa się nie liczy))

W warstwie ukrytej są neurony ukryte. Mogą być nieliniowe. Warstwa wyjściowa nie różni się niczym
od ukrytej poza tym, że jest jedna i wyjścia z niej są wyjściami z sieci.